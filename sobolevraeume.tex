% \datum{12. Oktober 2015}
\section{Sobolevräume}
\label{sec:sobolevraume}
\subsection{Der Raum der Testfunktionen}
Im Folgenden sei $\Omega \subseteq \R^{N}$ eine offene Menge (wenn nicht explizit etwas anderes gesagt wird). Wir verwenden folgende Funktionenräume (Vektorräume über $\R$ oder $\C$):
\begin{align*}
  C(\Omega) &= \set{u: \Omega \to \C: u \text{ ist stetig}}, \\
  L^{p}(\Omega) &= \set{u: \Omega \to \C: u \text{ ist messbar und } \int_{\Omega} \norm u^{p} < \infty}, \quad 1 \leq p < \infty
\end{align*}
versehen mit der Norm
\begin{align*}
  \nnorm{u}_{L^{p}} \coloneqq \( \int_{\Omega} \norm u^{p}\)^{\frac 1 p}
\end{align*}
und 
\begin{align*}
    L^{\infty}(\Omega) &= \set{u: \Omega \to \C: u \text{ ist messbar und } \exists C \geq 0: \set{\norm u > C} \text{ ist eine Nullmenge}}
\end{align*}
versehen mit der Norm
\begin{align*}
    \nnorm{u}_{L^{\infty}} \coloneqq \inf \set{C \geq 0:  \set{\norm u > C} \text{ ist Nullmenge}}
\end{align*}. 

Wir verwenden auch für $1 \leq p \leq \infty$ die Räume
\begin{align*}
  L^{p}_{loc}(\Omega) &= \set{u: \Omega \to \C: u \text{ ist messbar und }\forall K \subseteq \Omega \text{ kompakt ist } \int_{K}   \norm u^{p} < \infty}. 
\end{align*}
Das ist der Raum der lokal $p$-integrierbarern Funktionen. Es gilt für alle $p \in [1, \infty]$:
\begin{align*}
  L^{p}(\Omega) \subseteq L_{loc}^{p}(\Omega)\subseteq L_{loc}^{1}(\Omega)
\end{align*}
und
\begin{align*}
  C(\Omega) \subseteq L_{loc}^{1}(\Omega). 
\end{align*}
Für $u \in L^{1}_{loc}(\Omega)$ definieren wir den \markdef{Träger} durch
\begin{align*}
  \supp u\coloneqq \Omega \setminus \bigcup_{U_{i} \subseteq \Omega \text{ rel. offen}, u = 0 \text{ f.ü. auf } U_{i}} U_{i}
\end{align*}
(relativ abgeschlossen in $\Omega$, aber im Allgemeinen nicht im $\R^{N}$). 
Es gilt $L^{p}\(]0, 1[\) =L^{p}\([0, 1]\) \eqqcolon L^{p}\(0, 1\)$, aber $L_{loc}^{p}\(]0, 1[\) \neq L_{loc}^{p}\([0, 1]\)$! Es gilt $L^{p}(0, 1) = L_{loc}^{p}([0, 1])$. 

Eine Funktion $u \in L^{1}_{loc}(\Omega)$ hat \markdef{kompakten Träger}, wenn $\supp u$ kompakt ist.
\begin{align*}
  L_{c}^{p}(\Omega) & \coloneqq \set{ u \in L^{p}(\Omega): \supp u \text{ kompakt}}\\
  C_{c}(\Omega) & \coloneqq \set{ u \in C(\Omega): \supp u \text{ kompakt}}\\
  C_{c}^{k}(\Omega) & \coloneqq C_{c}(\Omega) \cap C^{k}(\Omega)\\
  C_{c}^{\infty}(\Omega) & \coloneqq C_{c}(\Omega) \cap C^{\infty}(\Omega). 
\end{align*}
Der Raum $C_{c}^{\infty}(\Omega)$ wird manchmal auch mit $\cD(\Omega)$ bezeichnet und heißt \markdef{Raum der Testfunktionen}. Die Elemente von $C_{C}^{\infty}$ heißen dementsprechend \markdef{Testfunktionen}. 
\begin{beispiel} Wichtige Testfunktion (einziges Beispiel):
  \begin{align*}
    u(x) =
    \begin{cases}
      e^{- \frac 1 {1- \nnorm x}^{2}}, & \text{falls } \nnorm x^{2} = \sum_{i = 1}^{N}x_{i}^{2} < 1\\
      0, & \text{sonst}. 
    \end{cases}
  \end{align*}
\end{beispiel}
Alle Testfunktionen sind integrierbar und haben ein positives Integral, also $C_{c}^{\infty} \subseteq L^{1}(\Omega)$. 
\begin{uebung}
  Für $u \in C(\Omega)$ gilt
  \begin{align*}
    \supp u = \cl \set{x \in \Omega: u(x) \neq 0}.
  \end{align*}
Der Abschluss ist bezüglich der relativen Topologie in $\Omega$ gemeint. 
\end{uebung}
Jede stetige Funktion ist auf einer kompakten Menge beschränkt. 
\begin{bemerkung}
  Manchmal wird $C_{c}^{\infty} = \cD(\Omega)$ auch mit $C_{0}^{\infty}(\Omega)$ (Was ist das? Schrecklich, Tod und Verderben!!!) bezeichnet. Achtung:
  \begin{align*}
    C_{0}\coloneqq \set{ u \in C(\Omega): \, \forall \eps > 0\, \exists K \subseteq \Omega \text{ kompakt }\forall x \in \Omega \setminus K: \norm{u(x)} \leq \eps}
  \end{align*}
\end{bemerkung}

\subsection{Die Faltung}
\begin{theorem} Young

Seien $f \in L^{1}(\R^{N})$, $g \in L^{p}(\R^{N})$, $p \in [1, \infty]$. Dann ist für fast alle $x \in \R^{N}$ die Funktion
\begin{align*}
  y \mapsto f(x - y)\cdot g(y)
\end{align*}
integrierbar. Wir definieren $f * g : \R^{N}\to \C$ durch
\begin{align*}
  (f * g)(x) \coloneqq \int_{\R^{N}} f(x - y)\cdot g(y) dy
\end{align*}
und nennen $f * g$ die \markdef{Faltung} von $f$ und $g$. Es gilt, dass $f*g \in L^{p}(\R^{N})$ und
\begin{align*}
  \nnorm {f * g}_{L^{p}} \leq   \nnorm {f}_{L^{1}} \cdot   \nnorm {g}_{L^{p}}. 
\end{align*}
\end{theorem}
\begin{beweis}
  \begin{enumerate}
  \item $p = 1$: Dann gilt mit Tonelli
    \begin{align*}
      \int_{\R^{N}}\int_{\R^{N}} \norm{f(x - y)\cdot g(y)} \, dy \, dx &= \int_{\R^{N}}\int_{\R^{N}} \norm{f(x - y)} \, dx \,  \cdot \norm{g(y)} \, dy  \\
      &= \int_{\R^{N}} \nnorm{f}_{L^{1}}\cdot g(y) \, dy  \\
      &= \nnorm{f}_{L^{1}}\cdot \nnorm{ g}_{L^{1}} < \infty  \\
    \end{align*}
nach Voraussetzung. Es gilt außerdem 
\begin{align*}
\nnorm{f*g}_{L^{1}} = \int_{R^{N}} \norm{f * g(x)} \, dx \leq \int_{\R^{N}}\int_{\R^{N}} \norm{f(x - y)\cdot g(y)} \, dy \, dx. 
\end{align*}
\item $1 < p < \infty$:
\begin{align*}
  g \in L^{p}(\R^{N}) \implies \norm g ^{p} \in L^{1}(\R^{N})
\end{align*}
Daraus folgt mit dem ersten Fall, dass für fast alle $x \in \R^{N}$
\begin{align*}
  y \mapsto \norm {f(x - y)}\cdot\norm{g(y)}^{p}
\end{align*}
integrierbar ist. Also ist für fast alle $x \in R^{N}$
\begin{align*}
    y \mapsto \norm {f(x - y)}^{\frac 1 p} \cdot\norm{g(y)}
\end{align*}
$p$-integrierbar. Außerdem gilt für $f \in L^{1} (\R^{N})$, dass $\norm f^{\frac 1 {p'}} \in L^{p'} \in \Omega$ für $p' = \frac p{p-1}$. Es folgt mit der Hölderschen Ungleichung
\begin{align*}
    y \mapsto \norm {f(x - y)}\cdot\norm{g(y)} = \norm {f(x - y)}^{\frac 1 {p'}}\cdot \norm {f(x - y)}^{\frac 1 {p}}\cdot  \norm {g(y)}
\end{align*}
ist integrierbar für fast alle $x \in \R^{N}$ und
\begin{align*}
  \norm {(f*g)(x)} &\leq \int_{R^{N}} \norm {f(x-y)} \norm{g(y)} dy \\
& \leq \(\int_{R^{N}} \norm {f(x-y)} dy\)^{\frac 1 {p'}} \cdot \( \int_{\R^{N}}\norm{f(x - y)}\cdot \norm{g(y)}^{p} dy\)^{\frac 1 p} \\
\implies \quad \int_{\R^{N}} \norm {f*p(x)}^{p} dx &\leq \nnorm f_{L^{1}}^{\frac p {p'}} \int_{R^{N}} \int_{\R^{N}} \norm {f(x-y)}  \cdot \norm{g(y)}^{p} dy \,dx\\
& =  \nnorm f_{L^{1}}^{\frac p {p'}} \int_{R^{N}} \int_{\R^{N}} \norm {f(x-y)} dx  \cdot \norm{g(y)}^{p} dy \\
& =  \nnorm f_{L^{1}}^{\frac p {p'} + 1} \nnorm g^{p}_{L^{p}}\\
& =  \nnorm f_{L^{1}}^{p} \nnorm g^{p}_{L^{p}}
\end{align*}

  \end{enumerate}
\item $p = \infty$ klar.
\end{beweis}

% \datum{13. Oktober 2015}

Variante:
\begin{korollar}
  Sei $f \in L^{p}_{loc}(\R^{N})$, $g \in L^{1}_{c}(\R^{N})$ (oder $f \in L^{1}_{loc}(\R^{N})$, $g \in L^{p}_{c}(\R^{N})$) für $1 \leq p \leq \infty$. Dann ist für fast alle $x \in \R^{N}$ die Funktion
  \begin{align*}
    y \mapsto f(x-y)\cdot g(y)
  \end{align*}
integrierbar auf $\R^{N}$. Die Funktion $f*g: \R^{N} \to \C$,
\begin{align*}
  (f*g)(x) = \int_{\R^{N}}f(x-y)g(y)dy
\end{align*}
ist damit wohldefiniert, heißt \markdef{Faltung} und es gilt $f*g \in L^{p}_{loc}(\R^{N})$.   
\end{korollar}
\begin{lemma} Träger einer Faltung

  Es gilt
  \begin{align*}
    \supp(f*g) \subseteq \cl(\supp f + \supp g)
  \end{align*}
(Träger ist abgeschlosen, Summe von zwei Mengen muss aber nicht abgeschlossen sein). 
\end{lemma}

\begin{beweis}
  Sei $x \notin \supp f + \supp g$. Dann ist $(x - \supp f) \cap \supp g = \emptyset$ (einfache Überlegung) und somit ist die Faltung von $f$ und $g$
  \begin{align*}
    (f*g)(x) &= \int_{\R^{N}}f(x-y)g(y) dy\\
&= \int_{\supp g \cap (x- \supp f)} f(x-y)g(y)dy = 0. 
  \end{align*}
\end{beweis}

\begin{lemma} Starke Stetigkeit der Shifts /Translationen
  
Für alle $p \in [1, \infty)$ und alle $f \in L^{P}(\R^{N})$ gilt
\begin{align*}
  \lim_{x \to 0} \int_{\R^{N}} \norm{f(x+y)- f(y)}^{p} dy = 0. 
\end{align*}
\end{lemma}
\begin{beweis}
  Für alle charakteristischen Funktionen $f = \1_{Q}$ mit $q = (a_{1}, b_{1})\times \dots \times (a_{N}, b_{N})$ folgt die Aussage aus dem Satz von Lebesgue (majorisierte Konvergenz). Damit ist die Aussage richtig für alle $f \in U$ mit
  \begin{align*}
    U = \Span \set{ \1_{Q}: \, Q = (a_{1}, b_{1}) \times \dots \times (a_{N}, b_{N}); a_{i}, b_{i} \in \R}. 
  \end{align*}
  Der Raum $U$ ist dicht in $L^{p}(\R^{N})$ (für alle $f \in L^{P}(\R^{N})$ und alle $\eps > 0$ gibt es ein $g \in U$ mit $\nnorm{f-g}_{L^{p}} < \eps$). Sei nun $f \in L^{p}(\R^{N})$, $\eps>0$. Sei $g \in U$ mit $\nnorm{f - g}_{L^{p}}  < \eps$. Dann gilt
  \begin{align*}
    \limsup_{x \to 0} \(\int_{R^{N}} \norm{f(x + y) - f(y)}^{p} dy\)^{\frac 1 p} &\leq  \limsup_{x \to 0} \(\int_{R^{N}} \norm{f(x + y) - g(x+y)}^{p} dy\)^{\frac 1 p} \\
& \, +  \(\int_{R^{N}} \norm{g(x + y) - g(y)}^{p} dy\)^{\frac 1 p}\\
& \, +  \(\int_{R^{N}} \norm{f(y) - g(y)}^{p} dy\)^{\frac 1 p} \\
&< 2 \eps.
  \end{align*}
Da $\eps > 0$ beliebig war, folgt die Aussage. 
\end{beweis}
\begin{theorem} Approximation der Eins
  
Sei $\phi \in L^{1}(\R^{N})$ positiv ($\phi(x) \geq 0$ f.ü.), $\int_{\R^{N}} \phi = 1$, und setze
\begin{align*}
  \phi_{n}(x) \coloneqq n^{N}\phi(n\cdot x), \quad x \in \R^{N}, \, n \in \N. 
\end{align*}
Dann gilt für alle $p \in [1, \infty)$ und alle $f \in L^{p}(\R^{N})$
\begin{align*}
  \lim_{n \to \infty} \nnorm{f * \phi_{n} - f}_{L^{p}} = 0. 
\end{align*}
Beobachtung: $\phi_{n} \geq 0$ und $\int_{\R^{N}} \phi_{n} = 1 = \norm {\phi_{n}}_{L^{1}}$. 
\end{theorem}
\begin{beweis}
  Sei zuerst $p = 1$, also $f \in L^{1}(\R^{N})$. Dann gilt
  \begin{align*}
    \nnorm{f*\phi_{n} - f}_{L^{1}(\R^{N})} &=  \int_{R^{N}} \norm{ \int_{\R^{N}} f(x - y)\phi_{n}(y) - f(x) \cdot 1 dy}\,   \,dx\\
 &\leq  \int_{R^{N}}  \int_{\R^{N}} \norm{f(x - y) - f(x)}\phi_{n}(y)\,  dy \,dx\\
 &=  \int_{R^{N}}  \int_{\R^{N}} \norm{f(x - \frac y n) - f(x)}\phi(y)\,  dy \,dx\\
 &=  \int_{R^{N}}  \int_{\R^{N}} \norm{f(x - \frac y n) - f(x)} dx \,\phi(y)\,  dy \to 0
  \end{align*}
für $n \to \infty$ nach Lebesgue. Dabei wurden verwendet: Ersetzen der Ein, Substitution, Tonelli.  
\end{beweis}
% einiges fehlt!
% \datum{19. Oktober 2015}

\begin{lemma}
  Für alle $X \subseteq \R^{N}$ kompakt und alle $\eps > 0$ existiert ein $\phi \in C_{c}^{\infty}(\R^{N})$ mit
  \begin{align*}
    0\leq \phi \phi 1 & \In \R^{N}\\
\phi = 1& \In K\\
\phi = 0& \in \R^{N}\setminus K^{\eps}. 
  \end{align*}
\end{lemma}
\begin{beweis}
  Sei $K \subseteq \R^{N}$ kompakt, $\eps > 0$. Sei $\psi \in C_{c}^{\infty}(\R^{N})$ die Testfunktion aus dem Beispiel oben, also
  \begin{align*}
    \psi(x) =
    \begin{cases}
      c \cdot e^{- \frac 1 {1 - \nnorm x^{2}}}, & \nnorm x < 1\\
      0, & \nnorm x \geq 1, 
    \end{cases}
  \end{align*}
wobei $c > 0$ so gewählt ist, dass $\int_{\R^{N}}\psi = 1$. Sei 
\begin{align*}
  \psi_{\eps}(x) = \frac 1 {\eps^{N}}\psi(\frac x \eps) \quad x \in \R^{N}.
\end{align*}
Dann gilt $\psi_{\eps} \in C_{c}^{\infty}(\R^{N})$, $\psi_{\eps}\geq 0$, $\int_{\R^{N}}\psi_{\eps} = 1$ und $\supp \psi_{\eps} = \cl(B(0, \eps))$. 

Sei
\begin{align*}
  \phi &\coloneqq \1_{K^{\eps}}* \psi_{\eps}\\
  \phi(x) &=  \int_{\R^{N}} \1_{K^{\eps}}(y)* \psi_{\eps}(x-y) dy\\
 &=  \int_{K^{\eps}} \psi_{\eps}(x-y) dy. 
\end{align*}
\begin{itemize}
\item Es gilt $\phi \geq 0$, da der Integrand $\psi_{\eps}\geq 0$ ist.
\item $\phi \leq 1$, da $\int_{K^{\eps}}\psi_{\eps}(x - y)\leq \int_{\R^{N}}\psi_{\eps}(x - y) = 1$. 
\item $\phi \in C^{\infty}(\R^{N})$ wegen der Glättungseigenschaften der Faltung. 
\item $\supp \phi \subseteq \supp \1_{K^{\eps}} + \supp \psi_{\eps} = \cl(K^{\eps}) + \cl(B(0, \eps)) = \cl (K^{2 \eps})$. 
Insbesondere ist $\supp \phi$ kompakt (und somit $\phi \in C_{c}^{\infty}(\R^{N})$) und $\phi = 0$ in $\R^{N} \setminus \cl(K^{2 \eps})$. 
\item Es bleibt noch $\phi = 1$ in $K$ zu zeigen: Für $x \in K$ ist
  \begin{align*}
    \phi(x) = \int_{K^{\eps}} \psi_{\eps}(x-y)dy = 1, 
  \end{align*}
da $x - \cl(B(0, \eps)) \subseteq K^{\eps}$ und $\int_{B(0, \eps)} \psi_{\eps} = \int_{\R^{N}}\psi_{\eps} = 1$. 
\end{itemize}
\end{beweis}

\begin{theorem}
  Für jede offene Menge $\Omega \subseteq \R^{N}$ und jedes $p \in [1, \infty)$ ist $C_{c}^{\infty}(\Omega)$ dicht in $L^{p}(\Omega)$. 
\end{theorem}
\begin{beweis}Technik: Abschneiden und glätten! Sei $f \in L^{p}(\Omega)$ und $\eps > 0$. 
  \begin{enumerate}
  \item Abschneiden: Wähle eine wachsende ($K_{n} \subseteq K_{n+1}$) Folge $(K_{n})$ von kompakten Mengen $K_{n} \subseteq \Omega$, sodass
    \begin{align*}
      \bigcup _{n \in \N} K_{n} = \Omega, 
    \end{align*}
zum Beispiel $K_{n} = \set{x \in \Omega: \dist(x, \Omega^{c})\geq \frac 1 n, \, \nnorm x \leq n}$. 

Wähle für alle $n \in \N$ eine Testfunktion $\psi_{n} \in C_{c}^{\infty}(\Omega)$ mit $0 \leq \psi_{n} \leq 1$ in $\Omega$ und $\psi_{n} = 1$ auf $K_{n}$ (das geht, siehe vorheriges Lemma). 

Dann gilt $\psi_{n} \to 1$ punktweise in $\Omega$ (für $n \to \infty$) und insbesondere $f\cdot \psi_{n} \to f$ punktweise in $\Omega$ und $\norm{f \cdot \psi_{n}} \leq \norm f$ in $\Omega$. Satz von Lebesgue:
\begin{align*}
  \nnorm{f\cdot \psi_{n} - f}_{L^{p}} \to 0
\end{align*}
für $n \to \infty$. Wähle $n \in \N$ so groß, dass mit $g_{1} = f\cdot \psi_{n} \in L^{p}(\Omega)$:
\begin{align*}
  \nnorm{f - g_{1}}_{L^{p}} \leq \frac \eps 2. 
\end{align*}
Beachte: Der Träger von $g_{1}$ in $\Omega$ ist kompakt. 

\item Glätten: Sei $\phi \in C_{c}^{\infty}(\R^{N})$ eine positive Testfunktion, sodass $\int_{\R^{N}}\phi = 1$ und $\supp \phi \subseteq \cl(B(0, 1))$. Setze dann
  \begin{align*}
    \phi_{m}(x) \coloneqq m^{N}(m\cdot x), \quad x \in \R^{N}, m \in \N
  \end{align*}
($\int_{\R^{N}} \phi_{m} = 1$, $\supp \phi_{m}\subseteq \cl(B(0, \frac 1 m))$ ). 

Das Theorem über die Approximation der Eins liefert
\begin{align*}
  \nnorm {g_{1}* \phi_{m} - g_{1}}_{L^{p}} \to 0 \quad (m \to \infty)
\end{align*}
(hier: $g_{1}$ durch $0$ auf $\R^{N}$ fortgesetzt). Glättungseigenschaften der Faltung:
\begin{align*}
  g_{1} * \phi_{m} \in C^{\infty}(\R^{N}). 
\end{align*}
Trägereigenschaften für Faltungen ($\supp \psi_{m}$ ist kompakt in $\Omega$):
\begin{align*}
  \supp g_{1}* \phi_{m} &\subseteq \supp g_{1} + \supp \phi_{m}\\
&\subseteq \supp \psi_{m} + B(0, \frac 1 m)\\
& \subseteq \Omega
\end{align*}
für $m$ groß genug. Also: Für genügend großes $m$ gilt $g_{1}* \phi_{m} \in C_{c}^{\infty}(\Omega)$ und $\nnorm{g_{1}*\phi_{m} - g_{1}}_{L^{p}} < \frac \eps 2$. Wähle nun $m$ groß genug und setze $g = g_{1} * \phi_{m} = (f\cdot \psi_{n})*\phi_{m}$. Dann ist $g \in C_{c}^{\infty}(\Omega)$ und $\nnorm{f - g}_{L^{p}} < \eps$. 
  \end{enumerate}
 \end{beweis}

Es wäre auch möglich gewesen, $g$ durch charakteristische Funktionen zu beschreiben:
\begin{align*}
 g =  (f\cdot \1_{K_{n}})* \phi_{m}. 
\end{align*}
\begin{theorem}Eindeutigkeit durch Testen 

Sei $\Omega \subseteq \R^{N}$ offen und $f \in L^{1}_{loc}(\Omega)$. Falls
\begin{align*}
  \int_{\Omega} f\cdot\phi = 0 \qquad \forall \phi \in C_{c}^{\infty}(\Omega), 
\end{align*}
dann ist $f = 0$ (fast überall). 
(Eventuell ist das namensgebend für 'Testfunktion'.)
\end{theorem}
\begin{beweis}
  Wir nehmen zuerst an, dass $f \in L^{1}(\Omega)$ ist. Für alle $\eps> 0$, wähle $g\coloneqq (f\cdot \psi_{n}) *\phi_{m}$ wie im Beweis des vorherigen Theorems, insbesondere ist $\nnorm{f - g}_{L^{1}} < \eps$, $g \in C_{c}^{\infty}(\Omega)$. Für alle $x \in K_{n}$ gilt:
  \begin{align*}
    g(x) = \int_{\R^{N}}f(y) \cdot\psi_{n}(y)\phi_{m}(x - y) dy = 0
  \end{align*}
nach Voraussetzung, da $\psi_{n}(\cdot) \cdot \phi_{m}(x - \cdot) \in C_{c}^{\infty}(\Omega)$. Da $n$ beliebig groß und $\eps$ beliebig klein gewählt werden kann, folgt $f = 0$ in $\Omega$. 
\vspace{5mm}

Falls $f$ nur lokal integrierbar auf $\Omega$ ist, dann wenden wir den ersten Schritt auf $f\cdot\psi$ mit $\psi \in C_{c}^{\infty}(\Omega)$ an. 
\end{beweis}

\subsection{Schwache Ableitungen und Sobolevräume}
\label{sec:schw-able-und}

Schreibweisen für partielle Ableitungen von differenzierbaren Funktionen:
\begin{align*}
  \frac{\partial u}{\partial x_{i}}, \, \partial_{x_{i}} u, \, \partial_{i} u
\end{align*}
(partielle Ableitung von $u$ nach $i$-ter Koordinate $x_{i}$). 

Für einen \markdef{Multiindex} $\alpha \in \N_{0}^{\N}$ und $x \in \R^{N}$ ist
\begin{align*}
  x^{\alpha}\coloneqq \prod_{i = 1}^{N} x_{i}^{\alpha_{i}}. 
\end{align*}
Für ein Polynom $P: \R^{N} \to \R$ und einen Multiindex $\alpha$,
\begin{align*}
  P(x) = \sum_{\norm \alpha\leq k} a_{\alpha}\cdot x^{\alpha}, \quad a_{\alpha} \in \C
\end{align*}
betrachten wir den partiellen Differentialoperator
\begin{align*}
  P(\partial) = \sum_{\alpha \in \N_{0}^{N}, \norm \alpha \leq k} a_{\alpha}\cdot \partial^{\alpha}
\end{align*}
mit $\partial^{\alpha} = \pd_{1}^{\alpha_{1}} \cdot \dots, \cdot \pd_{N}^{\alpha_{N}}$. Zum Beispiel für $P(x) = \nnorm x^{2} = \sum_{i = 1}^{N} x_{1}^{2}$ ist
\begin{align*}
  P(\pd) = \sum_{i = 1}^{N} \pd_{i}^{2} \eqqcolon \Delta
\end{align*}
der Laplaceoperator. 

Sei $f \in L^{1}_{loc}(\Omega)$, wobei $\Omega \subseteq \R^{N}$ eine offene Menge ist, und sei
\begin{align*}
  P(x) =  \sum_{\alpha \in \N_{0}^{N}, \norm \alpha \leq k} a_{\alpha}\cdot x^{\alpha}
\end{align*}
ein Polynom. Wir sagen, dass '$P(\pd)f \in L^{1}_{loc}(\Omega)$' (das macht eigentlich keinen Sinn!), falls es eine Funktion $g \in L^{1}_{loc}(\Omega)$ gibt, sodass
\begin{align*}
  \int_{\Omega}f P^{*}(\pd) \phi = \int_{\Omega} g\cdot\phi\\
P^{*}(\pd) \coloneqq \sum_{\alpha \in \N_{0}^{N}, \norm \alpha \leq k} (-a)^{\norm \alpha} a_{\alpha}\cdot \pd^{\alpha}
\end{align*}
für alle Testfunktionen $\phi \in C_{c}^{\infty}(\Omega)$ gilt. 

Beachte: Die Testfunktion $g$, wenn sie denn überhaupt existiert, ist eindeutig bestimmt. 
Sei nämlich $g_{1}, g_{2} \in L^{1}_{loc}(\Omega)$, sodass
\begin{align*}
  \forall \phi \in C_{c}^{\infty}(\Omega): \, (-1)^{\norm \alpha} \int_{\Omega} g_{1}\cdot\phi = \int_{\Omega} f \cdot P(\pd)\phi = (-1)^{\norm \alpha} \int_{\Omega} g_{2}\cdot\phi. 
\end{align*}
Dann ist $\int_{\Omega} (g_{1} - g_{2}) \cdot \phi = 0$ für alle $\phi \in C_{c}^{\infty}(\Omega)$ und somit (Eindeutigkeit durch Testen) $g_{1} = g_{2}$. Schreibweise:
\begin{align*}
  g \eqqcolon P(\pd)f. 
\end{align*}
Wir sagen, dass $f \in L^{1}_{loc}(\Omega)$ \markdef{$k$-mal schwach differenzierbar} ist ($k \in \N$), falls $\pd^{\alpha}f \in L^{1}_{loc}(\Omega)$ für alle Multiindizes $\alpha \in \N_{0}^{N}$ mit $\norm \alpha \leq k$. Die Funktion heißt \markdef{$\alpha$-te schwache Ableitung} von $f$. Aus dem Satz von Gauß (vereinfachte Version, das heißt, eine der beiden Funktionen im Satz von Gauß ist eine Testfunktion) folgt: 

Jede $k$-mal differenzierbare Funktion $f \in C^{k}(\Omega)$ ist $k$-mal schwach differenzierbar und $\alpha$-te klassische Ableitung (Analysis II), $\alpha$-te schwache Ableitung (Definition oben) stimmen überein. 

Gauß in der Version 'light': $f \in C^{1}(\Omega)$, $\phi \in C_{c}^{1}(\Omega)$, $j \in \set{1, \dots, N}$:
\begin{align*}
  \int_{\Omega} f\pd_{j}\phi = - \int_{\Omega}\pd_{j} f \cdot\phi. 
\end{align*}

%\datum{20. Oktober 2015}
Sei $\Omega \subseteq \R^{N}$ offen, $k \in \N$, $p \in [1, \infty]$. Wir definieren den \markdef{Sobolevraum}
\begin{align*}
  W^{k, p}(\Omega) \coloneqq \set{f \in L^{p}(\Omega): \, f \text{ ist } k\text{-mal schwach differentierbar und } \pd^{\alpha}f \in L^{p}(\Omega) \, \forall \alpha \in \N_{0}^{N}, \, \norm \alpha \leq k}
\end{align*}
und versehen diesen Raum mit der Norm
\begin{align*}
  \nnorm f_{W^{k ,p}} \coloneqq \(\sum_{\alpha \in \N_{0}^{N}, \norm \alpha \leq k} \nnorm {\pd^{\alpha} f}^{p}_{L^{p}}\)^{\frac 1p}, \quad p \in [1, \infty]
\end{align*}
beziehungsweise
\begin{align*}
    \nnorm f_{W^{k ,\infty}} \coloneqq \sup_{\alpha \in \N_{0}^{N}, \norm \alpha \leq k} \nnorm {\pd^{\alpha} f}_{L^{\infty}}. 
\end{align*}
Wir definieren außerdem die \markdef{Sobolevräume}
\begin{align*}
    W^{k, p}_{0}(\Omega) \coloneqq \cl_{\nnorm \cdot _{W^{k, p}}} (C_{c}^{\infty}(\Omega))
\end{align*}
(gemeint ist der Abschluss im Sinne der $W^{k, p}$-Norm) und setzen
\begin{align*}
  H^{k}(\Omega) &\coloneqq W^{k, 2}(\Omega)\\
  H^{k}_{0}(\Omega) &\coloneqq W^{k, 2}_{0}(\Omega).
\end{align*}
Die Räume $H^{k}(\Omega)$ und $H^{k}_{0}(\Omega)$ werden mit dem Skalarprodukt
\begin{align*}
  <f, g>_{H^{k}} \coloneqq \sum_{\alpha \in \N_{0}^{N}, \norm \alpha \leq k} \< \pd^{\alpha} f, \pd^{\alpha} g\>_{L^{2}}
\end{align*}
versehen, und sind damit Prä-Hilberträume. Die von $<\cdot, \cdot>_{K^{k}}$ induzierte Norm stimmt mir der Norm $\nnorm \cdot_{W^{k, 2}} \eqqcolon \nnorm \cdot _{H^{k}}$ überein. 
\begin{theorem}
  Die Sobolevräume $W^{k, p}(\Omega)$ und $W^{k, p}_{0}(\Omega)$ sind Banachräume. Sie sind separabel, falls $p \in [1, \infty)$ und reflexiv, falls $p \in (1, \infty)$. Die Räume $H^{k}(\Omega)$ und $H^{k}_{0}(\Omega)$ sind separable Hilberträume. 
\end{theorem}
\begin{beweis}
  Sei
  \begin{align*}
    A \coloneqq \set{\alpha \in \N_{0}^{N}: \, \norm \alpha \leq k}. 
  \end{align*}
Betrachte die Abbildung
\begin{align*}
  j : W^{k, p}(\Omega) \to L^{p}(\Omega)^{A}, \, f \mapsto (\pd^{\alpha} f)_{\alpha \in A}. 
\end{align*}
Wir versehen $L^{p}(\Omega)^{A}$ mit der Norm
\begin{align*}
  \nnorm {(f_{\alpha})_{\alpha \in A}}_{(L^{p})^{A}} \coloneqq \( \sum_{\alpha \in A} \nnorm {f_{\alpha}}_{L^{p}}^{p} \)^{\frac 1 p}, \quad p \in [0, \infty)
\end{align*}
beziehungsweise
\begin{align*}
    \nnorm {(f_{\alpha})_{\alpha \in A}}_{(L^{\infty})^{A}} \coloneqq \sup_{\alpha \in A} \nnorm{f_{\alpha}}_{L^{\infty}}
\end{align*}
Dann ist $j$ eine Isometrie und somit ist $W^{k, p}(\Omega)$ genau dann vollständig, wenn das Bild $j(W^{k, p}(\Omega))$ abgeschlossen in $L^{p}(\Omega)^{A}$ ist. Wir verwenden hier, dass $L^{p}(\Omega)$ und damit auch $L^{p}(\Omega)^{A}$ ein Banachraum ist.
Sei $(f_{n})$ eine Folge in $W^{k, p}(\Omega)$, sodass $(j(f_{n}))$ in $L^{p}(\Omega)^{A}$ gegen $(f_{\alpha})_{\alpha \in A}$ konvergiert. Für alle $n \in \N$, $\alpha \in A$, $ \phi \in C_{c}^{\infty}(\Omega)$ gilt
\begin{align*}
  \int_{\Omega} f_{n}\pd^{\alpha}\phi &= (-1)^{\norm \alpha} \int_{\Omega} \pd^{\alpha} f_{n} \phi\\
  \int_{\Omega} f_{(0, \dots, 0)}\pd^{\alpha}\phi &= (-1)^{\norm \alpha} \int_{\Omega} f_{\alpha} \phi \qquad \text{für } n \to \infty. 
\end{align*}
Aus der Definition der schwachen Ableitung folgt, dass $f_{(0, \dots, 0)}$ $\alpha$-mal schwach partiell differentierbar ist und $\pd^{\alpha}f_{(0, \dots, 0)} = f_{\alpha}$. Insbesondere ist $f_{(0, \dots, 0)}$ $k$-mal schwach differentierbar mit
\begin{align*}
  \pd^{\alpha}f_{(0, \dots, 0)} = f_{\alpha} \in L^{p}(\Omega), 
\end{align*}
das heißt, nach Definition von $W^{k, p}(\Omega)$
\begin{align*}
  f_{(0, \dots, 0)} \eqqcolon f \in W^{k, p}(\Omega)
\end{align*}
 beziehungsweise
 \begin{align*}
   (f_{\alpha})_{\alpha \in A} = (\pd^{\alpha} f)_{\alpha \in A} = j(f). 
 \end{align*}
Damit ist das Bild von $j$ abgeschlossen und $W^{k, p}(\Omega)$ vollständig. 
\vspace{5mm}

Falls $p \in [1, \infty)$, dann ist $L^{p}(\Omega)$ separabel, und somit ist $L^{p}(\Omega)^{A}$ separabel. Da Teilmengen von separablen Räumen separabel sind, ist $j(W^{k, p}(\Omega)) \stackrel \sim =  W^{k, p}(\Omega)$ separabel. 

\vspace{5mm}

Falls $p \in (1, \infty)$, dann ist $L^{p}(\Omega)$ reflexiv (siehe z.B. Brezis), und somit ist $L^{p}(\Omega)^{A}$ reflexiv (endliche Produkte von reflexiven Räumen sind reflexiv). Da abgeschlossene Unterräume von reflexiven Räumen reflexiv sind, ist $j(W^{k, p}(\Omega)) \stackrel \sim = W^{k, p}(\Omega)$ reflexiv. Die Aussagen über $H^{k}$ und $H^{k}_{0}$ folgen direkt daraus. 
\end{beweis}
\begin{theorem} Dichtheit der Testfunktionen in $W^{1, p}(\R^{N})$

Für alle $p \in [1, \infty)$ ist $C_{c}^{\infty}(\R^{N})$ dicht in $W^{1, p}(\R^{N})$. Insbesondere ist $W^{1, p}_{0}(\R^{N}) = W^{1, p}(\R^{N})$. 
\end{theorem}
Der Beweis des Theorems geht über 'Abschneiden und Glätten'. Man verwendet dabei die folgende Produktregel:
\begin{lemma} Produktregel
  
Für alle $f \in W^{1, p}(\Omega)$ ($p \in [1, \infty]$, $\Omega \subseteq \R^{N}$ offen) und alle $\phi \in C^{\infty}(\Omega) \cap W^{1, \infty} (\Omega)$ gilt
\begin{align*}
  f \cdot\phi \in W^{1, p}(\Omega)
\end{align*}
und
\begin{align*}
  \pd_{j}(f\cdot\phi) = (\pd_{j} f) \cdot\phi + f \cdot \pd_{j} \phi, \quad j \in \set{1, \dots, N}. 
\end{align*}
\end{lemma}
\begin{beweis}
  Für alle $\psi \in C_{c}^{\infty}(\Omega)$ gilt (mit der klassischen Produktregel, da $\phi, \psi \in C^{1}$):
  \begin{align*}
    \int_{\Omega}(f\cdot\phi)\cdot \pd_{j}\psi &= \int_{\Omega} f \cdot \pd_{j}(\underbrace{\phi\cdot\psi}_{\in C_{c}^{\infty}(\Omega)})- \int_{\Omega} f\cdot(\pd_{j}\phi)\cdot\psi\\
&= -\int_{\Omega} (\pd_{j}f)\cdot\phi\cdot\psi - \int_{\Omega} f\cdot(\pd_{j}\phi)\cdot\psi\\
&= -\int_{\Omega} \(\underbrace{\pd_{j}f)\cdot\phi + f\cdot(\pd_{j}\phi)}_{\in L^{p}(\Omega)}\)\cdot\psi\\
\implies f \cdot \phi \in W^{1, p}(\Omega) \, &\wedge \,  \pd_{j} (f \cdot\phi) = \pd_{j}f)\cdot\phi + f\cdot(\pd_{j}\phi). 
  \end{align*}
\end{beweis}

Notation: Sei $\Omega \subseteq \R^{N}$ (nicht mal offen). Wir schreiben
\begin{align*}
  \omega \Subset \Omega, 
\end{align*}
falls $\omega \subseteq \Omega$ und $\cl(\omega)$ (Abschluss in $\Omega$) kompakt ist. 
\begin{theorem} Friedrichs


  Seien $\Omega \subseteq \R^{N}$ offen, $p \in [1, \infty)$ und $f \in W^{1, p}(\Omega)$. Dann gibt es eine Folge $(f_{n})$ in $C_{c}^{\infty}(\R^{N})$, sodass
  \begin{align*}
    \left. f_{n}\right|_{\Omega} \to f \In L^{p}(\Omega)\\
    \left.\nabla f_{n}\right|_{\omega} \to \nabla f \In L^{p}(\omega)^{N} 
  \end{align*}
für alle $\omega \Subset \Omega$.
\end{theorem}
\begin{beweis}
  Abschneiden und glätten. 
\end{beweis}
\begin{bemerkung}
  \begin{enumerate}
  \item Nach einem Resultat von Meyers und Serin (1964), Titel '$H = W$', ist $W^{1, p}(\Omega) \cap C^{\infty}(\Omega)$ dicht in $W^{1, p}(\Omega)$, das heißt, für jedes $f \in W^{1, p}(\Omega)$ gibt es eine Folge $(f_{n}) \in W^{1, p}(\Omega) \cap C^{\infty}(\Omega)$, sodass
    \begin{align*}
      f_{n} \to f &\In L^{p}(\Omega), \\
      \nabla f_{n} \to \nabla f &\In L^{p}(\Omega).
    \end{align*}
\item Es gibt offene Mengen $\Omega \subseteq \R^{N}$, für die der Raum
  \begin{align*}
    \check C^{\infty}(\Omega) \coloneqq \set{u|_{\Omega}: \, u \in C_{c}^{\infty}(\R^{N})}
  \end{align*}
nicht dicht in $W^{1, p}(\Omega)$ ist. Die Frage der Dichtheit von $\check C^{\infty}(\Omega)$ in $W^{1, p}(\Omega)$ ist eine Frage der Regularität des Randes von $\Omega$. Ist zum Beispiel der Rand $\pd \Omega$ eine stetige Mannigfaltigkeit (der Dimenstion $N-1$), dann ist $\check C^{\infty}(\Omega)$ dicht in $W^{1, p}(\Omega)$ (siehe das 18. Internetseminar 2014/15). 
  \end{enumerate}
\end{bemerkung}

Sei $\Omega \subseteq \R^{N}$ offen, $f \in L^{i}_{loc}(\Omega)$, $h \in \R^{N}$. Wir definieren $\tau_{h}f$ durch
\begin{align*}
  (\tau_{h}f)(x) = f(x + h), \quad x \in \omega, 
\end{align*}
wobei $\omega \Subset \Omega$ und $\norm h < \dist(\omega, \pd \Omega)$, beziehungsweise, falls $\Omega = \R^{N}$, dann auch $\omega = \R^{N}$ und $h$ beliebig. 

\begin{theorem} Charakterisierung von $W^{1, p}$

Sei $\Omega\subseteq \R^{N}$ offen, $p \in (1, \infty]$ und $f \in L^{p}(\Omega)$. Dann sind folgende Aussafen äquivalent:
\begin{enumerate}
\item\label{num:1} $f \in W^{1, p}(\Omega)$,
\item\label{num:2} $\exists C\geq0 \forall \phi \in C_{c}^{\infty}(\Omega)\forall j \in \set{1, \dots, N}$:
  \begin{align*}
    \norm{\int_{\Omega} f \pd_{j}\phi} \leq C \nnorm{\phi}_{L^{p'}}
  \end{align*}
mit $p' = \frac{p}{p-1}$,
\item\label{num:3} $\exists C>0 \forall\omega \Subset \Omega\forall h \in \R^{N}$ mit $\norm h < \dist\set{\omega, \pd \Omega}$
  \begin{align*}
    \nnorm{\tau_{h}f-f}_{L^{p}(\omega)}\leq C\cdot\norm h. 
  \end{align*}
\end{enumerate}
\end{theorem}
\begin{beweis}
\numref{1}{2} folgt aus der Definition der schwachen Ableitungen $\pd_{j}f$ und Hölder. 

\numref{2}{1} Nach Voraussetzung ist das lineare Funktional
\begin{align*}
  L_{j}:& C_{c}^{\infty} \to \C\\
  &\phi \mapsto \int_{\Omega}f \pd_{j}\phi
\end{align*}
stetig bezüglich der $\nnorm\cdot_{L^{p'}}$-Norm auf $C_{c}^{\infty}(\Omega)$. Da $C_{c}^{\infty}(\Omega)$ dicht in $L^{p'}(\Omega)$ liegt ($p \neq 1$ und somit $p' \neq \infty$!!) besitzt $L_{j}$ eine eindeutige, stetige Fortsetzung auf $L^{p'}(\Omega)$. Also ist $L_{j} \in (L^{p'}(\Omega))'$. Da $(L^{p'}(\Omega))' \stackrel \sim = L^{p}(\Omega)$ (auch hier verwenden wir $p \neq 1$, für $p = 2$: Riez-Fréchet), gibt es ein $g_{j} \in L^{p}(\Omega)$ mit
\begin{align*}
  L_{j}(\phi) = \int_{\Omega}f \pd_{j}\phi \stackrel != -\int_{\Omega}g_{j}\phi. 
\end{align*}
 Also $g_{j} = \pd_{j}f \in L^{p}(\Omega)$, und da $j$ beliebig war: $f \in W^{1, p}(\Omega)$.

\ref{num:1} $\implies$\ref{num:3}: Sei $f \in C_{c}^{\infty}(\Omega)$, $\omega \Subset \Omega$, $h \in \R^{N}$, $\norm h < \dist(\omega, \pd \Omega)$. Dann gilt für alle $x \in \omega$ ($p > 1$):
\begin{align*}
  f(x + h) -f(x)= \int_{0}^{1}\frac d {dt} f(x + t\cdot h) dt
\end{align*}
und somit ($p < \infty$)
\begin{align*}
  \int_{\omega}\norm{f(x + h) -f(x)}^{p} dx &\leq \int_{\omega}\(\int_{0}^{1}\norm{\nabla f(x + t\cdot h) \cdot h}dt\)^{p}dx\\
 &\leq \int_{\omega} \int_{0}^{1}\norm{\nabla f(x + t\cdot h)}^{p} \cdot \norm h^{p} dt \,dx\\
 &\leq \norm h^{p}  \int_{0}^{1} \int_{\omega} \norm{\nabla f(x + t\cdot h)}^{p}  dt\\
 &\leq \norm h^{p}  \int_{0}^{1}  \nnorm{\nabla f}_{L^{p}(\omega')}^{p}  dt\\
 &= \norm h^{p} \nnorm{\nabla f}_{L^{p}(\omega')}^{p}
\end{align*}
wobei
\begin{align*}
  \omega' \coloneqq \set{x \in \R^{N}: \, \dist(x, \omega)< \norm h}, \quad \omega'\Subset \Omega.
\end{align*}
Falls $f \in W^{1, p}(\Omega)$ und $\omega, h$ und $\omega'$ wie oben, dann gibt es nach Friedrichs eine Folge $(f_{n})$ in $C_{c}^{\infty}(\Omega)$, sodass
\begin{align*}
  f_{n} \to f &\In L^{p}(\Omega), \\
  \nabla f_{n} \to \nabla f &\In L^{p}(\omega'),\\
\stackrel{n \to \infty}\implies \quad \nnorm{\tau_{h} f - f}_{L^{p} (\Omega)} &\leq \norm h \cdot \nnorm{\nabla f}_{L^{p}(\omega')}\\
&\leq \norm h \cdot \underbrace{\nnorm{\nabla f}_{L^{p}(\Omega)}}_{\eqqcolon c}. 
\end{align*}
Für $p = \infty$ folgt \ref{num:3} für $f \in C_{c}^{\infty}(\Omega)$ wieder aus dem HDI; verwende sodann Friedrichs in $W^{1, 1}$ und Konvergenz fast überall.  
\numref{3}{1} Sei $\phi \in C_{c}^{\infty}(\Omega)$, $j \in \set{1, \dots, N}$. Sei $\omega \Subset \Omega$ mit $\supp \phi \subseteq \omega$. Dann gilt
\begin{align*}
\int_{\Omega} f \pd_{j} \phi &= \lim_{t \to 0} \int_{\Omega} f \frac{\tau_{t_{j}}\phi - \phi} t\\
 &= \lim_{t \to 0} \int_{\Omega} f(x) \frac{\phi( x + \tau_{t_{j}}) - \phi(x)} t dx\\
 &= \lim_{t \to 0} \int_{\omega} f(x) \frac{f( x - \tau_{t_{j}}) - f(x)} t \phi(x) dx\\
 &= \lim_{t \to 0} \int_{\omega} f(x) \frac{\tau_{-t_{j}}f - f} t dx
\end{align*}
und somit nach Voraussetzung und Hölder
\begin{align*}
  \norm{\int_{\Omega} f \pd_{j} \phi}&\leq \limsup_{ t \to 0}\frac{\nnorm{\tau_{-t_{j}}f - f}_{L^{p}(\omega)}} t \cdot \nnorm \phi_{L^{p'}(\Omega)}\\
&\leq C \cdot \nnorm \phi_{L^{p'}(\Omega)}
\end{align*}
\end{beweis}

\begin{bemerkung}
  Im Fall $p = 1$ bleiben die Implikationen \ref{num:1} $\implies$ \ref{num:2} $\iff$ \ref{num:3} wahr. Der Raum
  \begin{align*}
 BV(\Omega)\coloneqq \set{f \in L^{1}(\Omega): \, \exists C\geq 0\, \forall \phi \in C_{c}^{\infty}(\Omega) \, \forall j \in \set{1, \dots, N}: \, \norm{\int_{\Omega} f \pd_{j}\phi} \leq C \cdot \nnorm \phi _{L^{\infty}}}
  \end{align*}
ist echt größer als $W^{1, 1}(\Omega)$. Elemente aus $BV(\Omega)$ heißen \markdef{Funktionen von beschränkter Variation}. ('Wahnsinnig spannender Raum'!)
\end{bemerkung}
\begin{theorem} Kettenregel 

Sei $\Omega\subseteq \R^{N}$ offen, $p \in [1, \infty]$. Sei $g \in C^{1}(\R)$, sodass $g(0) = 0$ und $\norm{g'}(x)\leq M$ für alle $x \in \R$. Dann ist für alle reellen $f \in W^{1, p}(\Omega)$ die Verknüpfung $ g\circ f \in W^{1, p}(\Omega)$ und die Ableitungen der Verknüpfung
\begin{align*}
  \pd_{j}(g \circ f) = (g' \circ f) \cdot \pd_{j} f. 
\end{align*}
\end{theorem}

\begin{beweis}
  Sei $f \in W^{1, p}(\Omega)\subseteq L^{p}(\Omega)$. Für alle $x \in \Omega$ gilt (mit dem Mittelwertssatz):
  \begin{align*}
    \norm{g(f(x))} &= \norm{g(f(x)) - g(0)} = \norm{g(f(x))}\\ 
    &\leq M\cdot\norm{f(x) - 0} \\
    &= M\cdot\norm{f(x)}\\
\implies \quad &g \circ f  \in L^{p}(\Omega)\\
\norm{(g'\circ f)(x) \pd_{j}f(x)}&\leq M\cdot \norm{\pd_{j}f(x)}\\
\implies \quad &(g' \circ f)\cdot \pd_{j}f  \in L^{p}(\Omega).
  \end{align*}
Sei $(f_{n})$ eine approximative Folge in $C_{c}^{\infty}(\Omega)$ wie im Satz von Friedrichs, das heißt
  \begin{align*}
     f_{n} \to f \In L^{p}(\Omega)\\
    \nabla f_{n}\to  \nabla f \In L^{p}(\omega), \quad \forall \omega \Subset \Omega.  
  \end{align*}
Sei $\phi \in C_{c}^{\infty}(\Omega)$. Dann gilt, für alle $j \in \set{1, \dots, N}$ und alle $n \in \N$
\begin{align*}
  \int_{\Omega} (g \circ f_{n})\pd_{j}\phi &= - \int_{\Omega} \pd_{j}(g \circ f_{n})\cdot\phi\\
 &= - \int_{\Omega} (g' \circ f_{n})\cdot\pd_{j} f_{n} \cdot\phi
\end{align*}
Satz von Lebesgue mehrfach angewendet und anschließend Definition eingesetzt: 
\begin{align*}
  \int_{\Omega} g \circ f \cdot \pd_{j}\phi &= - \int_{\Omega} (g' \circ f)\cdot\pd_{j} f \cdot\phi\\
g \circ f \in W^{1, p}(\Omega), \\
\pd_{j}(g \circ f) = (g' \circ f) \cdot \pd_{j} f. 
\end{align*}
\end{beweis}
Für reelle Funktionen $f, g : \Omega \to \R$ definieren wir punktweise
\begin{align*}
f \vee g &\coloneqq \sup \set{f, g}\\
f \wedge g &\coloneqq \inf \set{f, g}\\
f^{+}&\coloneqq f \vee 0\\
f^{-}&\coloneqq (-f) \vee 0\\
\norm f&\coloneqq f^{+} + f^{-}. 
\end{align*}
\begin{theorem}Verbandseigenschaften von $W^{1, p}(\Omega)$ ('$W^{1, p}$ ist noch ok')

Sei $\Omega\subseteq \R^{N}$ offen, $p \in [1, \infty]$, $f, g \in W^{1, ,p}(\Omega)$. Dann gilt
\begin{align*}
  f \vee g, f \wedge g, f^{+}, f^{-}, \norm f \in W^{1, p}(\Omega)
\end{align*}
und
\begin{align*}
  \pd_{j}f^{+} &= \1_{\set{f > 0}} \cdot \pd_{j}f,\\
  \pd_{j}f^{-} &= -\1_{\set{f < 0}} \cdot \pd_{j}f,\\
\pd_{j} \norm f &= \sgn f \cdot \pd_{j} f,\\
\pd_{j} (f \vee g) &= \1_{\set{f > g}} \cdot \pd_{j} f + \1_{f < g} \cdot \pd_{j} g,\\
\pd_{j} (f \wedge g) &= \1_{\set{f < g}} \cdot \pd_{j} f + \1_{f > g} \cdot \pd_{j} g. 
\end{align*}
\end{theorem}
\begin{beweis}
  Sei $g: \R \to \R$,
  \begin{align*}
    g(x) \coloneqq
    \begin{cases}
      x & x \geq 0\\ 0 & x < 0
    \end{cases}
  \end{align*}
und $g_{\eps}: \R\to \R$ ($\eps >0$):
\begin{align*}
   g_{\eps}(x) \coloneqq
    \begin{cases}
      x  - \frac \eps 2 & x \geq \eps,\\
      \frac 1{2\eps} x^{2} &0 < x <\eps\\
      0 & x \leq 0.
    \end{cases} 
\end{align*}
Dann gilt $g \circ f = f^{+}$ und $W^{+} \in W^{1, p}(\Omega)$ folgt dann aus $ g_{\eps}\circ f \in W^{1, p}(\Omega)$ und $\eps \to 0$ (und Kettenregel). Der Rest folgt aus 
\begin{align*}
  f^{-} &= (-f)^{+},\\
\norm f &= f^{+} + f^{-}, \\
f \vee g &= g + (f-g)^{+}, \\
f \wedge g &= f - (f - g)^{+}.
\end{align*}

\end{beweis}
$\check C^{\infty}(\Omega)$
$C_{c}^{\infty}(\Omega)$
$\sum_{\alpha \in \N_{0}^{N}, \norm \alpha \leq k}$
$W^{k, p}(\Omega)$


%%% Local Variables: 
%%% mode: latex
%%% TeX-master: "vorlesung"
%%% End: 
